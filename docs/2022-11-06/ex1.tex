% arara: lualatex: {
% arara: --> shell: yes,
% arara: --> interaction: batchmode
% arara: --> }

\documentclass[
	fontsize=10pt,
	paper=a4paper,
	DIV = calc,
	headinclude = false,
	footinclude = false,
	% parskip=half,
	% oneside,
	final
]{scrartcl}
\usepackage[top=1.5cm, bottom=1.5cm, left=1.5cm, right=1.5cm]{geometry}

\subject{\textcolor{DarkBlue}{Taller de Python para computación científica}}
\author{}
\date{}

\usepackage{mathtools}
\usepackage{enumerate}
\usepackage{graphicx}
\usepackage{pbox}
\usepackage{minted}
\usepackage{multicol}
\usepackage[svgnames]{xcolor}
\usepackage{lmodern}
\usepackage{microtype}
\usepackage[
	bookmarks,
	colorlinks
]{hyperref}

\titlehead{
	\pbox{\textwidth}{\textcolor{DarkBlue}{\bfseries\textsc{C++ Review DUNE}}}
	\hfill
	\includegraphics[height=0.2\textwidth]{../cppreviewdune.png}
}
\makeatletter
\renewcommand{\@maketitle}{%
	\clearpage
	\let\footnote\thanks
	\ifx\@extratitle\@empty \else
		\noindent\@extratitle \next@tpage \if@twoside \null\next@tpage \fi
	\fi
	\setparsizes{\z@}{\z@}{\z@\@plus 1fil}\par@updaterelative
	\ifx\@titlehead\@empty \else
		\begin{minipage}[t]{\textwidth}
			\@titlehead
		\end{minipage}\par
	\fi
	\null
	\vskip 2em%
	\begin{center}%
		\ifx\@subject\@empty \else
		{\subject@font \@subject \par}
		\vskip 1.5em
		\fi
		{\titlefont\huge \@title \par}%
		\vskip .5em
			{\ifx\@subtitle\@empty\else\usekomafont{subtitle}\@subtitle\par\fi}%
	\end{center}%
	\par
	\vskip 2em
}%
\makeatother

\title{\textcolor{DarkBlue}{Hoja de ejercicios 1}}
\subtitle{\color{DarkBlue}
  Revisión de Python:
  estructuras de control, repetición, 
  funciones integradas, funciones definidas por el usuario, NumPy
}

\begin{document}

\maketitle

\paragraph{\color{DarkBlue}Ejercicio 1.1}
¿Qué es un \emph{programa}?

\paragraph{\color{DarkBlue}Ejercicio 1.2}
Python es un \emph{lenguaje interpretado}.
¿Qué significa ``interpretado'' en este contexto?

\paragraph{\color{DarkBlue}Ejercicio 1.3}
Asignación:
\begin{listing}[ht!]
	\inputminted{python}{1.3.py}
\end{listing}
\begin{enumerate}[(a)]
	\item

	      Si ejecuta las tres líneas de código, ¿qué se imprimirá?
	      Explica tu respuesta.

	\item

	      Reescriba \mintinline{python}{my_int = my_int + 3} usando el
	      símbolo \mintinline{python}{+=}.
\end{enumerate}

\paragraph{\color{DarkBlue}Ejercicio 1.4}
Dada la expresión
\mintinline{python}{30 - 3**2 + 8 // 3**2 * 2 * 10}.

\begin{enumerate}[(a)]
	\item

	      ¿Cuál es el resultado de la expresión?

	\item

	      Según la precedencia y la asociatividad de los operadores en
	      Python, coloque correctamente entre paréntesis la expresión
	      de modo que obtenga el mismo resultado que el anterior.
\end{enumerate}

\paragraph{\color{DarkBlue}Ejercicio 1.5}
Escriba un programa en Python que acepte un número $n$ y
calcule el valor de
\begin{math}
	n +
	n\cdot n +
	n\cdot n\cdot n +
	n\cdot n\cdot n\cdot n
\end{math}.

\paragraph{\color{DarkBlue}Ejercicio 1.6}
Considere un \emph{triángulo} con lados de longitud $3$u, $7$u y
$9$u.
La \emph{ley de cosenos} establece que dados los tres lados de un
triángulo $a$, $b$ y $c$, el ángulo $C$ entre los lados $a$ y $b$ es
\begin{math}
	c^{2}=
	a^{2}+
	b^{2}-
	2ab\cos\left(C\right)
\end{math}.
Escriba un programa en Python para calcular los tres ángulos en el
triángulo.

\paragraph{\color{DarkBlue}Ejercicio 1.7}
El radio y la masa de la Tierra son
\begin{math}
	r =
	6378\times10^{3}
\end{math}
metros y
\begin{math}
	m_{1}=
	5.9742\times10^{24}
\end{math}
kilogramos, respectivamente.
Pedro tiene una masa de $X$ kilogramos.
Pida al usuario que ingrese $X$ y luego calcule la fuerza
gravitacional $\left(F\right)$ y la aceleración debida a la gravedad
$\left(g\right)$ causada por la fuerza gravitacional ejercida sobre
él por la Tierra.
Recuerde, $F=G\frac{m_{1}m_{2}}{r^{2}}$ y $F=mg$.
Sea la constante de gravitación universal $G=6.67300\times 10^{-11}$
(en unidades de m$^{3}$kg$^{-1}$s$^{-2}$).
Comprueba que el valor resultante de $g$ es cercano a 9.8 ms$^{-2}$.

\paragraph{\color{DarkBlue}Ejercicio 1.8}
(Usando módulos) Python viene con \emph{cientos de módulos}.
Aquí hay un desafío para ti: encuentre un módulo que pueda importar
que genere la fecha de hoy para que pueda imprimirlo.
Use su motor de búsqueda favorito para obtener ayuda para encontrar
qué módulo necesita y cómo usarlo.
Al final, su tarea es hacer lo siguiente:
\begin{listing}[ht!]
	\inputminted{python}{1.8.py}
\end{listing}

\paragraph{\color{DarkBlue}Ejercicio 1.9}
¿Cuántos números naturales menores que $N$ son divisibles por 2 y no
por 3?
Escriba un programa para imprimirlos.
\emph{Sugerencia}:
\begin{listing}[ht!]
	\inputminted{python}{1.9.py}
\end{listing}

\paragraph{\color{DarkBlue}Ejercicio 1.10}
¿Qué es un \emph{iterador}?
Dé dos ejemplos de iteradores.

\paragraph{\color{DarkBlue}Ejercicio 1.11}
Prediga el resultado dado \mintinline{python}{x = 0},
\mintinline{python}{y = 2}, \mintinline{python}{z = 4}.

\begin{multicols}{2}
	\begin{enumerate}[(a)]
		\item

		      \mintinline{python}{x or y}

		\item

		      \mintinline{python}{x and y}

		\item

		      \mintinline{python}{x or z}

		\item

		      \mintinline{python}{(x and y) or (y and z)}
	\end{enumerate}
\end{multicols}

\paragraph{\color{DarkBlue}Ejercicio 1.12}
Prediga el resultado de cada línea:
\begin{listing}[ht!]
	\inputminted{python}{1.12.py}
\end{listing}

\paragraph{\color{DarkBlue}Ejercicio 1.13}
Escriba un programa que imprima todos los años bisiestos desde 2000
hasta 2099 (inclusive).
\emph{Sugerencia}:
\begin{listing}[ht!]
	\inputminted{python}{1.13.py}
\end{listing}

\paragraph{\color{DarkBlue}Ejercicio 1.14}
Complete la siguiente tabla con valores \mintinline{python}{True} o
\mintinline{python}{False}: un valor en cada cuadro vacío.

\begin{table}[ht!]
	\centering
	\begin{tabular}{|c|c|c|c|c|c|c|}
		\hline
		\mintinline{python}{p}     & \mintinline{python}{q}
		                           & \mintinline{python}{(not p) or q}   & \mintinline{python}{(p and q) or q}
		                           & \mintinline{python}{(p or q) and p} & \mintinline{python}{(p or q) and (p and q)} \\
		\hline
		\mintinline{python}{True}  & \mintinline{python}{True}
		                           &                                     &
		                           &                                     &                                             \\
		\hline
		\mintinline{python}{True}  & \mintinline{python}{False}
		                           &                                     &
		                           &                                     &                                             \\
		\hline
		\mintinline{python}{False} & \mintinline{python}{True}
		                           &                                     &
		                           &                                     &                                             \\
		\hline
		\mintinline{python}{False} & \mintinline{python}{False}
		                           &                                     &
		                           &                                     &                                             \\
		\hline
	\end{tabular}
\end{table}

\paragraph{\color{DarkBlue}Ejercicio 1.15}
La \emph{fórmula cuadrática} que calcula las raíces para una ecuación
cuadrática $ax^{2}+bx+c=0$ es
\begin{math}
	x=
	\frac{-b\pm\sqrt{b^{2}-4ac}}{2a}
\end{math}.
Debido a que la raíz cuadrática de un negativo es imaginaria, se
puede usar la expresión de la raíz cuadrada (conocida como el
\emph{discriminante}) para verificar el tipo de raíz.
Si el discriminante es negativo, las raíces son imaginarias.
Si el discriminante es cero, solo hay una raíz.
Si el discriminante es positivo, hay dos raíces.

\begin{enumerate}[(a)]
	\item

	      Escriba un programa que use la fórmula cuadrática para
	      generar las raíces reales, es decir, ignore las raíces
	      imaginarias.
	      Usa el discriminante para determinar si hay una o dos
	      raíces y luego imprime la respuesta apropiada.

	\item

	      Python usa la letra \mintinline{python}{j} para representar
	      el número imaginario $i$ (una convención utilizada en
	      ingeniería eléctrica).
	      Sin embargo, la \mintinline{python}{j} de Python siempre debe
	      ir precedida de un número.
	      Es decir, \mintinline{python}{1j} es equivalente a la $i$
	      matemática.
	      Implemente el manejo de raíces imaginarias a su programa.
\end{enumerate}

\paragraph{\color{DarkBlue}Ejercicio 1.16}
Para entender lo que un programa intenta lograr, es esencial poder
seguir el flujo de control.
En el siguiente ejemplo, ¿qué sucede cuando \mintinline{python}{x = 4}?
\emph{Sugerencia}: visualize el programa en
\url{https://pythontutor.com}.

\begin{listing}[ht!]
	\inputminted{python}{1.16.py}
\end{listing}

\begin{enumerate}[(a)]
	\item

	      El programa sale del bucle \mintinline{python}{while} y deja
	      de ejecutarse.

	\item

	      El programa sale del ciclo \mintinline{python}{for}, pero la
	      condición \mintinline{python}{while} sigue siendo verdadera,
	      lo que da como resultado un bucle infinito.

	\item

	      El programa no se interrumpe, simplemente continúa procesando
	      el bucle \mintinline{python}{for}.
\end{enumerate}

\paragraph{\color{DarkBlue}Ejercicio 1.17}
Escribe un algoritmo para freír un huevo.
Pruébalo con un amigo(a): en una cocina, lee el algoritmo y haz que
el amigo(a) haga exactamente lo que dices. ¿Cómo lo hizo?

\paragraph{\color{DarkBlue}Ejercicio 1.18}

Dada la cadena \mintinline{python}{m = "Monty Python"}:

\begin{enumerate}[(a)]
	\item

	      Escribe una expresión para imprimir el primer carácter.

	\item

	      Escriba una expresión para imprimir el último carácter.

	\item

	      Escriba una expresión que imprima
	      \mintinline{console}{Python}.
\end{enumerate}

\paragraph{\color{DarkBlue}Ejercicio 1.19}
Dada \mintinline{python}{s = "Earth"}, ¿qué devuelve
\mintinline{python}{s.strip()}, \mintinline{python}{s.lstrip()} y
\mintinline{python}{s.rstrip()}?

\paragraph{\color{DarkBlue}Ejercicio 1.20}
¿Qué hace esta función? ¿Qué devuelve para
\mintinline{python}{number = 5}?

\begin{listing}[ht!]
	\inputminted{python}{1.20.py}
\end{listing}

\paragraph{\color{DarkBlue}Ejercicio 1.21}
¿Qué hace esta función? ¿Qué devuelve si \mintinline{python}{x = 5}?

\begin{listing}[ht!]
	\inputminted{python}{1.21.py}
\end{listing}

\paragraph{\color{DarkBlue}Ejercicio 1.22}
¿Qué hace esta función? ¿Qué número devuelve esta función?

\begin{listing}[ht!]
	\inputminted{python}{1.22.py}
\end{listing}

\paragraph{\color{DarkBlue}Ejercicio 1.23}
Escribe una función que tome la masa como entrada y devuelva su
energía equivalente.
$\left(E = mc^{2}.\right)$

\paragraph{\color{DarkBlue}Ejercicio 1.24}
La \emph{sucesión de Fibonacci} es $1$, $1$, $2$, $3$, $5$, $8$,
$13$, \ldots.
Puedes ver que el primer y el segundo número son ambos $1$.
A partir de entonces, cada número es la suma de los dos números
anteriores.
\begin{enumerate}[(a)]
	\item

	      Escribe una función para imprimir los primeros $N$ números
	      de la sucesión de Fibonacci.

	\item

	      Escribe una función para imprimir el número $N$ de la
	      secuencia.

\end{enumerate}

\emph{Sugerencia}:

\url{https://docs.python.org/es/3/tutorial/introduction.html#first-steps-towards-programming}

\paragraph{\color{DarkBlue}Ejercicio 1.25}
Has decidido programar una calculadora sencilla.
Su calculadora puede realizar las siguientes operaciones:
\begin{multicols}{2}
	\begin{enumerate}[(a)]
		\item

		      Puede sumar dos números.
		\item

		      Puede restar un número de otro.

		\item

		      Puede multiplicar dos números.

		\item

		      Puede dividir un número por otro.
	\end{enumerate}
\end{multicols}

Escriba una función que le pida al usuario dos números y la operación
que desea realizar en estos dos enteros, y devuelva el resultado.
\emph{Nota}: Las opciones deben ingresarse como cadenas.

\paragraph{\color{DarkBlue}Ejercicio 1.26}
Escriba una función que acepte una lista de cadenas y las ordene
alfabéticamente.

\emph{Sugerencia}:
\begin{listing}[ht!]
	\inputminted{python}{1.26.py}
\end{listing}

\paragraph{\color{DarkBlue}Ejercicio 1.27}
¿Qué es NumPy?

\paragraph{\color{DarkBlue}Ejercicio 1.28}
¿Qué es un arreglo y en qué se diferencia de una lista?
¿Cuál es el nombre de la clase de arreglo integrada en NumPy?

\paragraph{\color{DarkBlue}Ejercicio 1.29}
Cree los siguientes arreglos NumPy:

\begin{enumerate}[(a)]
	\item

	      Un arreglo $1$-D llamado \mintinline{python}{ceros} teniendo
	      10 elementos y todos los elementos son igual a cero.

	\item

	      Un arreglo $1$-D llamado \mintinline{python}{vocales}
	      teniendo los elementos \mintinline{python}{"a"},
	      \mintinline{python}{"e"}, \mintinline{python}{"i"},
	      \mintinline{python}{"o"} y \mintinline{python}{"u"}.

	\item

	      Un arreglo $2$-D llamado \mintinline{python}{unos} teniendo
	      filas y 5 columnas y todo los elementos son igual a uno y
	      \mintinline{python}{dtype = int}.

	\item

	      Use listas anidadas de Python para crear un arreglo 2-D
	      llamado \mintinline{python}{myarray1} que tenga 3 filas y 3
	      columnas y almacene los siguientes datos
	      \begin{equation*}
		      \begin{bmatrix}
			      3.7   & -1   & -14 \\
			      0     & 2.24 & 1.9 \\
			      -10.1 & 1    & 3
		      \end{bmatrix}.
	      \end{equation*}

	\item

	      Un arreglo $2$-D llamado \mintinline{python}{myarray2} usando
	      \mintinline{python}{np.arange()} que tiene 3 filas y 5
	      columnas con valor \mintinline{python}{start = 4},
	      \mintinline{python}{step = 4} y
	      \mintinline{python}{dtype=float}.
\end{enumerate}

\paragraph{\color{DarkBlue}Ejercicio 1.30}
Usando los arreglos creados en el \textbf{Ejercicio 1.29},
escribe comandos NumPy para lo siguiente:

\begin{enumerate}[(a)]
	\item

	      Encuentre las \emph{dimensiones}, la \emph{forma}, el
	      \emph{tamaño}, el tipo de dato de los elementos y el tamaño
	      de los elementos de los arreglos \mintinline{python}{ceros},
	      \mintinline{python}{vocales}, \mintinline{python}{unos},
	      \mintinline{python}{myarray1} y
	      \mintinline{python}{myarray2}.

	\item

	      Reagrupe el arreglo de \mintinline{python}{unos} para tener
	      los 10 elementos en una sola fila.

	\item

	      Muestre el segundo y tercer elemento del arreglo
	      \mintinline{python}{vocales}.

	\item

	      Muestre los elementos de la segunda y tercera fila del
	      arreglo \mintinline{python}{myarray1}.

	\item

	      Muestre los elementos de la primera y segunda columna del
	      arreglo \mintinline{python}{myarray1}.

	\item

	      Muestre los elementos de la primera columna de la segunda y
	      tercera fila del arreglo \mintinline{python}{myarray1}.

	\item

	      Invierta el arreglo \mintinline{python}{vocales}.
\end{enumerate}

\paragraph{\color{DarkBlue}Ejercicio 1.31}
(Operaciones aritméticas)
Usando los arreglos creados en el \textbf{Ejercicio 1.29},
escribe comandos NumPy para lo siguiente:

\begin{enumerate}[(a)]
	\item

	      Divide todos los elementos del arreglo
	      \mintinline{python}{unos} por 3.

	\item

	      Sume los arreglos \mintinline{python}{myarray1} y
	      \mintinline{python}{myarray2}.

	\item

	      Reste \mintinline{python}{myarray1} de
	      \mintinline{python}{myarray2} y almacene el resultado en un
	      nuevo arreglo.

	\item

	      Multiplique \mintinline{python}{myarray1} y
	      \mintinline{python}{myarray2} elemento por elemento.

	\item

	      Realice la multiplicación de arreglos
	      \mintinline{python}{myarray1} y \mintinline{python}{myarray2}
	      y almacene el resultado en un nuev arreglo
	      \mintinline{python}{myarray3}.

	\item

	      Divide \mintinline{python}{myarray1} por
	      \mintinline{python}{myarray2}.

	\item

	      Encuentre el cubo de todos los elementos de
	      \mintinline{python}{myarray1} y divida el arreglo resultante
	      por 2.

	\item

	      Encuentre la raíz cuadrada de todos los elementos de
	      \mintinline{python}{myarray2} y divida el arreglo resultante
	      por 2.
	      El resultado debe redondearse a dos decimales.
\end{enumerate}

\paragraph{\color{DarkBlue}Ejercicio 1.32}
Usando los arreglos creados en el \textbf{Ejercicio 1.29},
escribe comandos NumPy para lo siguiente:

\begin{enumerate}[(a)]
	\item

	      Encuentra el arreglo transpuesto de \mintinline{python}{unos}
	      y \mintinline{python}{myarray2}.

	\item

	      Ordena el arreglo \mintinline{python}{vocales} al revés.

	\item

	      Ordene el arreglo \mintinline{python}{myarray1} de modo que
	      traiga el menor valor de la columna en la primera fila y así
	      sucesivamente.
\end{enumerate}

\paragraph{\color{DarkBlue}Ejercicio 1.33}
Usando los arreglos creados en el \textbf{Ejercicio 1.29},
escribe comandos NumPy para lo siguiente:

\begin{enumerate}[(a)]
	\item

	      Use \mintinline{python}{np.split()} para dividir el arreglo
	      \mintinline{python}{myarray2} en 5 arreglos en forma de
	      columna.
	      Almacene sus arreglos resultantes en
	      \mintinline{python}{myarray2A},
	      \mintinline{python}{myarray2B},
	      \mintinline{python}{myarray2C},
	      \mintinline{python}{myarray2D} y
	      \mintinline{python}{myarray2E}.
	      Imprima los arreglos \mintinline{python}{myarray2A},
	      \mintinline{python}{myarray2B},
	      \mintinline{python}{myarray2C},
	      \mintinline{python}{myarray2D} y
	      \mintinline{python}{myarray2E}.

	\item

	      Divida el arreglo \mintinline{python}{ceros} de la matriz
	      en el índice del arreglo 2, 5, 7, 8 y almacene los arreglos
	      resultantes en \mintinline{python}{cerosA},
	      \mintinline{python}{cerosB}, \mintinline{python}{cerosC} y
	      \mintinline{python}{cerosD} e imprímalas.

	\item

	      Concatene los arreglos \mintinline{python}{myarray2A},
	      \mintinline{python}{myarray2B} y
	      \mintinline{python}{myarray2C} en un arreglo que tenga 3
	      filas y 3 columnas.
\end{enumerate}

\paragraph{\color{DarkBlue}Ejercicio 1.34}

Cree un arreglo $2$-D llamado \mintinline{python}{myarray4} usando
\mintinline{python}{np.arange()} que tenga 14 filas y 3 columnas con
\mintinline{python}{start = -1}, \mintinline{python}{step = 0.25}.
Divida este arreglo por filas en 3 partes iguales e imprima el
resultado.

\paragraph{\color{DarkBlue}Ejercicio 1.35}
Usando \mintinline{python}{myarray4} creado en la pregunta de arriba,
escribe comandos para lo siguiente:
\begin{enumerate}[(a)]
	\item

	      Encuentra la suma de todos los elementos.

	\item

	      Encuentra la suma de todos los elementos por filas

	\item

	      Encuentra la suma de todos los elementos por columnas.

	\item

	      Encuentra el máximo de todos los elementos.

	\item

	      Encuentre el mínimo de todos los elementos en cada fila.

	\item

	      Encuentra la media de todos los elementos en cada fila.

	\item

	      Encuentre la desviación estándar por columnas.
\end{enumerate}
\emph{Sugerencia}: \url{https://ncert.nic.in/textbook/pdf/keip106.pdf}
\end{document}